\documentclass{article}

% Packages

\usepackage{fullpage}
\usepackage{amsmath, amsthm, amsfonts, amssymb, mathtools, calrsfs, tensor, physics,tikz-cd}
\usepackage[mathscr]{euscript}
\usepackage{graphicx}
\graphicspath{ {images/} }
\usepackage{enumitem}
\setlist[description]{font=\normalfont}

% Custom Commands

\newcommand*\diff{\mathop{}\!\mathrm{d}}
\newcommand*\Diff[1]{\mathop{}\!\mathrm{d^#1}}
\newcommand*\nrml{\vartriangleleft}
\newcommand*\scr[1]{\mathscr{#1}}
\newcommand*\bb[1]{\mathbb{#1}}
\newcommand*\la{\langle}
\newcommand*\ra{\rangle}
\newcommand*\gen[1]{\langle #1 \rangle}
\newcommand*\x{\times}
\newcommand*\st{\text{ s.t. }}
\newcommand*\ord[1]{\left\vert#1\right\vert}
\newcommand*\aut{\text{Aut}}
\newcommand*\lcm{\text{lcm}}
\newcommand*\mcal{\mathcal}
\newcommand*\es{\emptyset}
\newcommand*\im{\text{ Im }}
\newcommand*\N{\mathbb N}
\newcommand*\Z{\mathbb Z}
\newcommand*\R{\mathbb R}
\newcommand*\Q{\mathbb Q}
\newcommand*\C{\mathbb C}
\newcommand*\te[1]{\text{#1}}
\newcommand*\en[1]{\begin{enumerate}#1\end{enumerate}}
\newcommand*\e{\varepsilon}
\newcommand*\p[1]{\left(#1\right)}
\newcommand*\ps[1]{\left[#1\right]}
\newcommand*\pc[1]{\left\{#1\right\}}
\newcommand*\f[2]{\frac{#1}{#2}}
\newcommand*\mat[2]{\left(\begin{array}{#1}#2\end{array}\right)}
\newcommand*\ocross{\otimes}
\newcommand*\I{\te{i}}
\newcommand*\pd[3]{\frac{\partial^{#3} #1}{\partial {#2}^{#3}}}
\newcommand*\td[3]{\frac{d^{#3}#1}{d #2^{#3}}}
\newcommand*\m{\te{Mat}}
\newcommand*\End{\te{End}}
\newcommand*\irr{\te{Irr}}
\newcommand*\sgn{\te{sgn}}
\newcommand*\pn[2]{\left\|#1\right\|_{#2}}
\newcommand*\esssup{\te{ess sup}}
\newcommand*\essinf{\te{ess inf}}

% Miscellaneous

\newtheorem{theorem}{Theorem}
\usetikzlibrary{matrix,arrows,decorations.pathmorphing}

% Title
\title{Current Progress}
\date{\today}

\begin{document}
\maketitle

We have the following Fourier transform
\begin{align}
\phi(\vec x,t)&=\int\f{\diff^3 k}{(2\pi)^3}\Phi(\vec k,\omega)e^{\i(\vec k\cdot\vec x-\omega t)}\,.
\end{align}
The wave equation
\begin{align}
\alpha^2\nabla^2\phi=\partial_t^2\phi\,,
\end{align}
imposes the dispersion relation $\alpha k=\omega$ from which we obtain the solution
\begin{align}
\phi(\vec x,t)&=\int\f{\diff^3 k}{(2\pi)^3}\Phi(\vec k)e^{\I(\vec k\cdot\vec x-\alpha k t)}\,.
\end{align}
We can now impose the constraint (where $K$ is the bulk modulus)
\begin{align}
-p(\vec x,0)&=K\nabla^2\phi(\vec x,0)\,,
\end{align}
which, in Fourier space, reads as
\begin{align}
P(\vec k)&=K k^2\Phi(\vec k)\,.
\end{align}
From this, we obtain the displacement field's Fourier coefficients
\begin{align}
U(\vec k)=\I\f{P(\vec k)}{K}\f{\vec k}{k^2}\,,
\end{align}
and hence
\begin{align}
u(\vec x,t)=\int\f{\diff^3 k}{(2\pi)^3}\I\f{P(\vec k)}{K}\f{\vec k}{k^2}e^{\I(\vec k\cdot\vec x-\alpha k t)}\,.
\end{align}
From this, we can obtain the total energy
\begin{align}
E&=\f12\int_V\diff^3 x\p{\rho\abs{\partial_t u}^2+(\lambda+2\mu)\abs{\nabla\cdot u}^2}\,,\\
&=\f{(\lambda+2\mu)}{K^2}\int\f{\diff^3 k}{(2\pi)^3}\abs{P(\vec k)}^2\,,\\
&=\f{(\lambda+2\mu)}{K^2}\int\f{\diff k}{(2\pi)^3}4\pi k^2\abs{P(k)}^2\,,\\
&=4\pi\f{(\lambda+2\mu)}{K^2}\int\f{\diff \tilde\lambda}{\tilde\lambda^4}\abs{P(k)}^2\,.
\end{align}
We now treat an overpressure of $p_0$ confined to a cylinder of radius $r_x$ and height $h$.
\begin{align}
P(\vec k)&=\f{4\pi r_x p_0}{k^2\cos\theta\sin\theta} J_1(k\sin\theta r_x)\sin\p{\f{h k\cos\theta}{2}}\,,\\
&\sim_{k\to0}\f{2\pi r_x p_0}{k^2\cos\theta\sin\theta} k\sin\theta r_x \sin\p{\f{h k\cos\theta}{2}}\,,\\
&=\f{2\pi r_x^2 p_0}{k\cos\theta}  \sin\p{\f{h k\cos\theta}{2}}\,.
\end{align}
Hence
\begin{align}
E_{\te{large }\lambda_0}&=\f{(\lambda+2\mu)}{K^2}\int_{0}^{k_0}\int_{0}^{\pi}\int_{0}^{2\pi}\f{\diff^3 k}{(2\pi)^3}\abs{\f{2\pi r_x^2 p_0}{k\cos\theta}  \sin\p{\f{h k\cos\theta}{2}}}^2\,,\\
&=(r_x^2 p_0)^2\f{(\lambda+2\mu)}{K^2}\int_{0}^{k_0}\int_{0}^{\pi}\diff k\diff\theta \sin\theta\ps{\f{1}{\cos\theta}  \sin\p{\f{h k\cos\theta}{2}}}^2\,,\\
&=\ps{(r_x^2 p_0)^2\f{(\lambda+2\mu)}{K^2}}\ps{\f{hk_0\cos(hk_0)+\sin(hk_0)+hk_0(-2+hk_0\te{Si}(hk_0))}{2h}}\,,\\
&=\ps{(r_x^2 p_0)^2\f{(\lambda+2\mu)}{K^2}}\ps{\f{\pi }{\lambda_0}\cos(\f{2\pi h}{\lambda_0})+\f{1}{2h}\sin(\f{2\pi h}{\lambda_0})+\f{\pi }{\lambda_0}\p{-2+\f{2\pi h}{\lambda_0}\te{Si}\p{\f{2\pi h}{\lambda_0}}}}\,,\\
&\approx_{h\gg\lambda_0\gg1}\ps{(\sigma_x p_0)^2\f{(\lambda+2\mu)}{K^2}}\ps{\f{\pi h}{\lambda_0^2}}
\end{align}
The real question lies in the factor of $(\sigma_xp_0)$ which I believe is proportional to, if not equal to $\td{E}{x}{}$. This result, however, has the feature that it only has one power of $h$, which means that it goes as the energy (if things work as I think they do). Whence, the fraction of energy deposited in the low frequency spectrum is
\begin{align}
\Xi=\pi\abs{\td{E}{x}{}}\f{\lambda+2\mu}{K^2\lambda_0^2}\,,
\end{align}
where we have taken $E_{\te{total}}=\sigma_x h p_0$ and $\abs{\td{E}{x}{}}=\sigma_x p_0$.
\\\\
Our questions are: does this make sense, i.e. can $\Xi$ be greater than 1? It seems as though $\Xi$ is almost proportional to the energy deposition squared. If so, where did we go wrong? If not, how should we proceed? We had a discussion with Professor Taylor the other day, and he was very concerned that we are considering only the linear regime. Clearly a shockwave will form, but will the shockwave to be relegated to the high frequency spectrum?  Moreover, Prof. Taylor brought up the importance of the geometry of the projectile. A pointed tip will push matter to the side, while a blunt tip will force matter along its trajectory. Does this matter for such high velocities? I suppose this would correspond to the fraction of energy deposited in the shear versus compressional modes.
\end{document}