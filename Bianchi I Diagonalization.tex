\documentclass{article}

% Packages

\usepackage{fullpage}
\usepackage{amsmath, amsthm, amsfonts, amssymb, mathtools, calrsfs, tensor, physics,tikz-cd}
\usepackage[mathscr]{euscript}
\usepackage{graphicx}
\graphicspath{ {images/} }
\usepackage{enumitem}
\usepackage[T1]{fontenc}
\usepackage[latin9]{inputenc}
\usepackage{amsmath}
\setlist[description]{font=\normalfont}

% Custom Commands

\newcommand*\diff{\mathop{}\!\mathrm{d}}
\newcommand*\Diff[1]{\mathop{}\!\mathrm{d^#1}}
\newcommand*\nrml{\vartriangleleft}
\newcommand*\scr[1]{\mathscr{#1}}
\newcommand*\bb[1]{\mathbb{#1}}
\newcommand*\la{\langle}
\newcommand*\ra{\rangle}
\newcommand*\gen[1]{\langle #1 \rangle}
\newcommand*\x{\times}
\newcommand*\st{\text{ s.t. }}
\newcommand*\ord[1]{\left\vert#1\right\vert}
\newcommand*\aut{\text{Aut}}
\newcommand*\lcm{\text{lcm}}
\newcommand*\mcal{\mathcal}
\newcommand*\es{\emptyset}
\newcommand*\im{\text{ Im }}
\newcommand*\N{\mathbb N}
\newcommand*\Z{\mathbb Z}
\newcommand*\R{\mathbb R}
\newcommand*\Q{\mathbb Q}
\newcommand*\C{\mathbb C}
\newcommand*\te[1]{\text{#1}}
\newcommand*\en[1]{\begin{enumerate}#1\end{enumerate}}
\newcommand*\e{\varepsilon}
\newcommand*\p[1]{\left(#1\right)}
\newcommand*\ps[1]{\left[#1\right]}
\newcommand*\pc[1]{\left\{#1\right\}}
\newcommand*\f[2]{\frac{#1}{#2}}
\newcommand*\mat[2]{\left(\begin{array}{#1}#2\end{array}\right)}
\newcommand*\ocross{\otimes}
\newcommand*\I{\te{i}}
\newcommand*\pd[3]{\frac{\partial^{#3} #1}{\partial {#2}^{#3}}}
\newcommand*\td[3]{\frac{d^{#3}#1}{d #2^{#3}}}
\newcommand*\m{\te{Mat}}
\newcommand*\End{\te{End}}
% Miscellaneous

\newtheorem{theorem}{Theorem}
\usetikzlibrary{matrix,arrows,decorations.pathmorphing}

% Title
\title{B1-Perturbations stage 2}
\date{\today}

\begin{document}
\maketitle
First, some definitions. It is important that we all use the same
conventions. We should stick to the $\left(-,+,+,+\right)$ signature,
which is not what Weinberg does. The less minus signs we use, the
less we are prone to sign mistakes.

The background metric is
\[
ds^{2}=S^{2}\left[-d\eta^{2}+\gamma_{ij}(\eta)dx^{i}dx^{j}\right]\,,
\]
where
\[
\gamma_{ij}:=\mbox{diag}(e^{2\beta_{1}},e^{2\beta_{2}},e^{2\beta_{3}})\,,\qquad\sum_{i=1}^{3}\beta_{i}=0\,.
\]
Note that we are working in a very specific coordinate system in which
the shear $\sigma_{ij}:=(\gamma_{ij})'/2$ has only two independent
components. Since $\sigma_{ij}$ is a symmetric traceless matrix,
the other three componentes can be seen as the three Euler angles
needed to rotate $\gamma_{ij}$ to a general coordinate system. 

The Lie derivative of the background metric along the vector $\xi$
is
\begin{align*}
{\cal L}_{\xi}\bar{g}_{00} & =-2S^{2}\left(T'+HT\right)\\
{\cal L}_{\xi}\bar{g}_{0i} & =S^{2}\left(-\partial_{i}T+\gamma_{ij}\partial_{0}\xi^{j}\right)\\
{\cal L}_{\xi}\bar{g}_{ij} & =S^{2}\left(2{\cal H}T\gamma_{ij}+2T\sigma_{ij}+2\partial_{(i}\xi_{j)}\right)
\end{align*}
This is always true, regardless of the splitting. 

We will parameterize $\xi^{\mu}$ as
\[
\xi^{\mu}=\left(T,\partial^{1}X,\partial^{2}Y,\partial^{3}Z\right)
\]
and the line element as
\[
ds^{2}=S^{2}\left[-(1+2A)d\eta^{2}+2B_{i}dx^{i}d\eta+(\gamma_{ij}+h_{ij})dx^{i}dx^{j}\right]\,,
\]
where
\[
B_{i}=\left(\partial_{1}E,\partial_{2}F,\partial_{3}G\right)
\]
and $h_{ij}$ will be built later. Gauge transformations are such
that
\[
\Delta\delta g_{\mu\nu}={\cal L}_{\xi}\bar{g}_{\mu\nu}
\]
It is thus straightforward to compute the following transformation
for $A$ and $B_{i}$: 
\begin{align*}
A & \rightarrow A+T'+{\cal H}T=A+(ST)'/S\\
E & \rightarrow E-T+X'-2\beta_{1}'X=E-T+(\gamma^{11}X)'/\gamma^{11}\\
F & \rightarrow F-T+Y'-2\beta_{2}'Y=F-T+(\gamma^{22}Y)'/\gamma^{22}\\
G & \rightarrow G-T+Z'-2\beta_{3}'Z=G-T+(\gamma^{33}Z)'/\gamma^{33}
\end{align*}


In order to parameterize $h_{ij}$, we need ask how many ways there
exist to build a tensor from scalars only (since we want SSS decomposition).
There are two ways: either by multiplying a scalar by $\gamma_{ij}$,
or by taking two derivatives of a scalar. Thus we write
\[
h_{ij}=\left(\gamma_{ij}+\frac{\sigma_{ij}}{{\cal H}}\right)2C+\bar{h}_{ij}
\]
where $C$ is a scalar and $\bar{h}_{ij}$ is a traceless matrix built
out of (two) derivatives of 5 new scalar fields. One possibility is
\[
\bar{h}_{ij}=\left(\begin{array}{ccc}
2\partial_{1}^{2}B & \partial_{1}\partial_{2}H & \partial_{1}\partial_{3}I\\
 & 2\partial_{2}^{2}Q & \partial_{2}\partial_{3}J\\
 &  & 2\partial_{3}^{2}D
\end{array}\right)\,,
\]
with the constraint
\[
\partial_{1}^{2}B+\partial_{2}^{2}Q+\partial_{3}^{2}D=0\,.
\]


The gauge transformations are
\begin{align*}
C & \rightarrow C+{\cal H}T\\
B & \rightarrow B+X\\
Q & \rightarrow Q+Y\\
D & \rightarrow D+Z\\
H & \rightarrow H+X+Y\\
I & \rightarrow I+X+Z\\
J & \rightarrow J+Y+Z
\end{align*}

The following are Gauge invariant combinations:
\begin{align}
&A+\f1S\ps{S\p{E-\f{(\gamma^{11} B)'}{\gamma^{11}}}}'\,,\\
&A+\f1S\ps{S\p{F-\f{(\gamma^{22} Q)'}{\gamma^{22}}}}'\,,\\
&A+\f1S\ps{S\p{G-\f{(\gamma^{33} D)'}{\gamma^{33}}}}'\,,\\
&C+\scr H\ps{E-\f{(\gamma^{11}B)'}{\gamma^{11}}}\,,\\
&C+\scr H\ps{F-\f{(\gamma^{22}Q)'}{\gamma^{22}}}\,,\\
&C+\scr H\ps{G-\f{(\gamma^{33}D)'}{\gamma^{33}}}\,,\\
&H-B-Q\,,\\
&I-B-D\,,\\
&J-D-Q\,,\\
&I+J+K-2(B+Q+D)\,,\\
&H+I-J-2B\,,\\
&H+J-I-2Q\,,\\
&I+J-H-2D\,,
\end{align}
and so on...
\\\\
<<<<<<< HEAD
%In analogy to The Theory of Cosmological Perturbations in an Anisotropic Universe, we pick the 6 GIV's:
%\begin{align}
%\Phi&=A+\f1S\ps{S\p{E-\f{(\gamma^{11} B)'}{\gamma^{11}}}}'\,,\\
%\f{\Psi}{\scr H}=\Phi_1&=E+\f{C}{\scr H}-\f{(\gamma^{11}B)'}{\gamma^{11}}\,,\\
%\Phi_2&=F+\f{C}{\scr H}-\f{(\gamma^{22}Q)'}{\gamma^{22}}\,,\\
%\Phi_3&=G+\f{C}{\scr H}-\f{(\gamma^{33}D)'}{\gamma^{33}}\,,\\
%\Xi_1&=H-B-Q\,,\\
%\Xi_2&=I-B-D\,,\\
%\Xi_3&=J-D-Q\,.
%\end{align}
%Perhaps a good choice of Gauge is $E=B=D=Q=0$. In this case we get
%\begin{align}
%\Phi&=A\,,\\
%\Psi&=C\,,\\
%\Phi_2&=F+\f{C}{\scr H}\,,\\
%\Phi_3&=G+\f{C}{\scr H}\,,\\
%\Xi_1&=H\,,\\
%\Xi_2&=I\,,\\
%\Xi_3&=J\,.
%\end{align}
%Another good choice may be $C=B=D=Q=0$. We then have
%\begin{align}
%\Phi&=A+\f1S\p{SE}'\,,\\
%\Phi_1&=E\,,\\
%\Phi_2&=F\,,\\
%\Phi_3&=G\,,\\
%\Xi_1&=H\,,\\
%\Xi_2&=I\,,\\
%\Xi_3&=J\,.
%\end{align}
Maybe we want to choose the off diagonal components of $\bar h_{ij}$ to be 0. In this case, appropriate GIV's may be
\begin{align}
\varphi&=A+\p{\f{C}{\scr H}}'-C\,,\\
\xi_1&=E+\f{C}{\scr H}-\f12\f{(\gamma^{11}(H+I-J))'}{\gamma^{11}}\,,\\
\xi_2&=F+\f{C}{\scr H}-\f12\f{(\gamma^{22}(H-I+J))'}{\gamma^{22}}\,,\\
\xi_3&=G+\f{C}{\scr H}-\f12\f{(\gamma^{33}(-H+I+J))'}{\gamma^{33}}\,,\\
\zeta_1&=B-\f12(H+I-J)\,,\\
\zeta_2&=Q-\f12(H-I+J)\,,\\
\zeta_3&=D-\f12(-H+I+J)\,.
We should check the following: there exists a choice of gauge such that $C=I=J=K=0$. Then our 7 variables (which only constitute 6 degrees of freedom via the constraint $\partial_{1}^{2}B+\partial_{2}^{2}Q+\partial_{3}^{2}D=0$) are
\begin{align}
\varphi&=A\,,\\
\xi_1&=E\,,\\
\xi_2&=F\,,\\
\xi_3&=G\,,\\
\zeta_1&=B\,,\\
\zeta_2&=Q\,,\\
\zeta_3&=D\,.
\end{align}
\end{document}
